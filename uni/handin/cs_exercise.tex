% Computer science, Datastructures & Algorithms handin with pseudocode samples
% Test
\documentclass{article}

\usepackage[utf8]{inputenc}
\usepackage[ngerman]{babel}
\usepackage{amsmath}

% Pseudocode
\usepackage{algorithm}
\usepackage{algpseudocode}

\setlength{\parindent}{0in}

\newcommand{\uebungsGruppe}{1}
\newcommand{\zettelNummer}{1}
\newcommand{\studierenderEins}{Rick Sanchez (71-5097-2)}
\newcommand{\studierenderZwei}{Morty Smith  (C-137)}
\newcommand{\studierenderDrei}{Squanchy (X)}

\newcounter{AufgabenCounter}
\setcounter{AufgabenCounter}{1}
\newcounter{TeilaufgabenCounter}
\newenvironment{aufgabe}{\section*{Aufgabe \theAufgabenCounter}\setcounter{TeilaufgabenCounter}{1}}{\stepcounter{AufgabenCounter}}
\newenvironment{teilaufgabe}{\paragraph*{\alph{TeilaufgabenCounter})}}{\stepcounter{TeilaufgabenCounter}}

\renewcommand{\to}{\textnormal{ to }}
\newcommand{\bigO}{\mathcal{O}}

\begin{document}

\title{Datenstrukturen und Algorithmen \\ Heimübung \zettelNummer{}}
\author{\studierenderEins{} \\
    \studierenderZwei{} \\
    \studierenderDrei{}}

\maketitle

% Aufgabe 1
\begin{aufgabe}

    % Teilaufgabe a)
    \begin{teilaufgabe}

        Wir haben den folgenden Algorithmus entworfen:

        \begin{algorithm}[H]
            \caption{FindeGrößtesElementFor($A[1, \dots, n]$)}
            \begin{algorithmic}[1]
                \State Ergebnis $\gets 0$ 
                \For{$i \gets 1 \to n$} 
                    \If{$A[i] \geq$ Ergebnis}
                        \State Ergebnis $\gets A[i]$
                    \EndIf
                \EndFor
                \State \textbf{Return} Ergebnis
            \end{algorithmic}
        \end{algorithm}

        Alternativ haben wir den folgenden äquivalenten Algorithmus entworfen:

        \begin{algorithm}[H]
            \caption{FindeGrößtesElementFor($A[1, \dots, n]$)}
            \begin{algorithmic}[1]
                \State Ergebnis $\gets 0$ 
                \State $i \gets 1$
                \While{$i \leq n$} 
                    \If{$A[i] \geq$ Ergebnis}
                        \State Ergebnis $\gets A[i]$
                    \EndIf
                    \State $i \gets i+1$
                \EndWhile
                \State \textbf{Return} Ergebnis
            \end{algorithmic}
        \end{algorithm}

    \end{teilaufgabe}

    % Teilaufgabe b)
    \begin{teilaufgabe}
        Im ersten Algorithmus haben wir eine Zuweisung in Zeile $1$ und in Zeilen $3-4$ der For-Schleife einen Vergleich und eine Zuweisung. 
        Die For-Schleife wird $n$ mal durchlaufen.
        Damit ergibt sich eine Gesamtlaufzeit von $\bigO(n)$.

        Der zweite Algorithmus hat $2$ Zuweisungen außerhalb der While-Schleife und $2$ Zuweisungen, einen Vergleich und eine Addition innerhalb der While-Schleife in Zeilen $4-7$.
        Da auch die While-Schleife maximal $n$ mal durchlaufen wird, ergibt sich eine Laufzeit von $\bigO(n)$.
    \end{teilaufgabe}

\end{aufgabe}

% Aufgabe 2
\begin{aufgabe}

    % Teilaufgabe a)
    \begin{teilaufgabe}
        Es gilt ...
    \end{teilaufgabe}

    % Teilaufgabe b)
    \begin{teilaufgabe}
        Hier können wir das in der Vorlesung gezeigte Theorem anwenden und erhalten ...
    \end{teilaufgabe}

    % Teilaufgabe c)
    \begin{teilaufgabe}
        Ein Gegenbeispiel konstruiert sich wie folgt. ...
    \end{teilaufgabe}

\end{aufgabe}

% Aufgabe 3
\begin{aufgabe}
    Für Sortieren haben wir gelernt, dass ...
\end{aufgabe}

\end{document}
